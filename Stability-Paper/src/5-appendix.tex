\appendix

\section{Appendix}\label{appendix}

\dfInequality*
\begin{proof}
    We will prove this by induction on $d$: when $d=f$, the inequality holds with equality. Assuming that the statement is valid for $d$, let us verify it for  $d+1$. Denoting the left and right sides of the inequality as $l(d, f)$ and $r(d)$, respectively, we have
    \begin{equation*}
        \begin{split}
            r(d+1) - l(d+1, f) 
            & \geq \left(r(d+1) - r(d)\right) - \left(l(d+1, f) - l(d, f)\right) \\
            & = \left(d2^d + 2^{d+1} + 2\right) - \left(d+f+2\right)\left(2^{d-1}+2^{d-f}\right) \\
            & = 2^{d-f}\left(d-f+2\right)\left(2^{f-1}-1-\frac{2f}{d-f+2}\right)+2 \\
            & \geq 2^{d-f}\left(d-f+2\right)\left(2^{f-1}-1-f\right)
        \end{split}
    \end{equation*}
    The obtained expression is non-negative for $f>2$. For $f=2$ and $d\geq4$, we have $2^{f-1}-1-\frac{2f}{d-f+2}\geq0$, and for $f=2$ and $d=2,3$, the initial inequality can be checked explicitly.    
\end{proof}

\sufficientFractionLemma*
\begin{proof}
    We start by claiming that
    \begin{equation}\label{binom78}
        \forall n > 2,\,j\in \mathbb{Z}:\: \binom{n}{j-1}+\binom{n}{j}+\binom{n}{j+1} \leq \frac{7}{8}\cdot 2^{n}.
     \end{equation}
    This can be checked by induction on $n$: \eqref{binom78} holds for $n=3$, and assuming it holds for $n-1$, we have
    \[
        \tbinom{n}{j-1}+\tbinom{n}{j}+\tbinom{n}{j+1} = \Bigl(\tbinom{n-1}{j-2}+\tbinom{n-1}{j-1}+\tbinom{n-1}{j}\Bigr) + \Bigl(\tbinom{n-1}{j-1}+\tbinom{n-1}{j}+\tbinom{n-1}{j+1}\Bigr) \leq \tfrac{7}{8}\cdot 2^{n-1} + \tfrac{7}{8}\cdot 2^{n-1} = \tfrac{7}{8}\cdot 2^n.
    \]
    Denote $P = S_2 \setminus S_1$ and $Q = S_1 \setminus S_2$, with $q=|Q|$ and $p=|P|\geq 2$. Let us also denote 
    \[
        \mathcal{D}_j = \{T\subseteq P\cup Q: |T\cap P|-|T\cap Q|=j\}\text{ for an integer }j.
    \] 
    Clearly, $|S\cap S_2| - |S\cap S_1| = |S\cap P| - |S\cap Q|$, and it is sufficient to show that the family $\mathcal{D}_{-1}\cup \mathcal{D}_{0}\cup \mathcal{D}_{1}$ contains at most $\frac{7}{8}\cdot 2^{p+q}$ sets. This is obvious if $p=2$ and $q=0$, so we will further assume $p+q>2$. To any set $T\in \mathcal{D}_j$ we may assign the set $(T\cap P)\cup(Q\setminus T)$ of size $q+j$. Such assignment constitutes a bijection between $\mathcal{D}_j$ and $(q+j)$-subsets of $P\cup Q$. Thus, $|\mathcal{D}_j|=\binom{p+q}{q+j}$, and with \eqref{binom78} we conclude
    \[
        |\mathcal{D}_{-1}\cup \mathcal{D}_{0}\cup \mathcal{D}_{1}| = \binom{p+q}{q-1}+\binom{p+q}{q}+\binom{p+q}{q+1}\leq \frac{7}{8}\cdot 2^{p+q}.
    \]
\end{proof}


\smallDifferenceLemma*
\begin{proof}
    The  statement is trivial for $d=2$, so in what follows we assume $d>2$. Then $|\sss|>2$ and clearly $\sss$ contains sets of at most two different sizes (that differ by one). Let $U, V \in \sss$ both be of size $k\in [d-2]$. Observe that there are now only four options for sets in $\sss$:
    \begin{enumerate}[(a)]
        \item $U \cup V$ of size $k+1$. \label{cup}
        \item Sets of size $k$ that are contained in $U \cup V$. \label{cupSubset}
        \item Sets of size $k$ that contain $U\cap V$ as a subset. \label{capSupset}
        \item $U \cap V$ of size $k-1$. \label{cap}
    \end{enumerate}
Since $|(U \cup V)\setminus (U \cap V)|=2$, the sets \ref{cup} and \ref{cap} cannot occur simultaneously. Similarly, if sets $B, C$ satisfy \ref{cupSubset}, \ref{capSupset}, respectively, and both differ from $U$ and $V$, then $|B\setminus C|=2$. Thus, with the exception of $U$ and $V$, the sets \ref{cupSubset} and \ref{capSupset} cannot be present together. There are $k+1$ and $d-k$ sets satisfying \ref{cupSubset} and \ref{capSupset}, respectively, so $|\sss|=d$ is only possible if $k=d-2$ or $k=1$ with $\sss = \{S\subseteq[d-1]: |S| \geq d-2\}$ or $\sss = \{S\subseteq[d-1]: |S| \leq 1\}$, respectively. 
\end{proof}

\crossPolyRepresentation*
\begin{proof}
    Let $\{e_i\}$ be the standard basis of $\R^d$ and consider the linear transform that takes $\{a_1, \ldots, a_{d-1}, s\}$ to $\{e_d + e_1, \ldots, e_d + e_{d-1}, 2e_d\}$. This transform and a translation by $-e_d$ maps $P$ to the standard cross-polytope 
    \[
        K = \operatorname{Conv}\Bigl(\{e_1, \ldots, e_d\}\cup\{-e_1, \ldots, -e_d\}\Bigr).
    \]
\end{proof}

\noindent We finish with a conjecture that generalises Theorem~\ref{d_plus_one_two_d} and Theorem~\ref{d2d_plus_2d}:

\begin{conjecture}\label{generalisation}
    Let $\aaa,\bb \subseteq \R^d$ be families of vectors that both linearly span $\R^d$. Suppose that $\la a, b\ra \in \{0,1\}$ holds for all $a \in \aaa$, $b \in \bb$. Furthermore, suppose that $|\aaa|$ and $|\bb|$ are both strictly larger than $2^{k-1}(d-k+2)$ for some $k\in [0,d]$. Then $\left|\mathcal{A}\right| \cdot\left|\mathcal{B}\right| \leq (2^{d-k}+k)2^k(d-k+1)$.
\end{conjecture}
The motivating example for this conjecture is the following generalisation of Example~\ref{cubeOctop}:
\begin{example}\label{generalCubeOctop}
    Let $\{e_i\}$ be the standard basis of $\R^d$, $k\in [0,d]$,
    \begin{align*}
        \aaa= & \left\{\sum_{i=k+1}^{d}\delta_i e_i\right\}\cup\left\{e_1,\ldots,e_k\right\},\,\bb=\left\{\sum_{i=1}^{k}\delta_i e_i + e_j\right\}\cup\left\{\sum_{i=1}^{k}\delta_i e_i\right\}, \\
        & \text{ where } \delta_i\text{ range over }\left\{0, 1\right\} \text{ and }j\text{ over }[k+1,d].
    \end{align*}
    Here, $|\aaa|=2^{d-k}+k$ and $|\bb|=2^k (d-k+1)$.
\end{example}
Our enumeration of distinct sets with binary scalar products in dimensions up to 4, where `distinct' refers to a lack of linear isomorphism, supports Conjecture~\ref{generalisation}. The conjecture also holds for all sets emerging from 2-level polytopes in dimensions up to 8.
