\section{Preliminaries}\label{secStability}
\paragraph{Notation.}
In what follows, we will often treat vectors in $\R^d$ as points in an affine space, with $\operatorname{dim}$ always referring to the affine dimension while $\operatorname{span}$ referring to linear span. The set of integers  from $1$ to $n$ is denoted $[n]$. 

Let $\aaa, \bb$ be families of vectors that both linearly span $\R^d$ and have binary scalar products, that is, $\la a, b \ra \in \{0,1\}$ for all $a \in \aaa$ and $b \in \bb$. We will use the following two simple observations a few times throughout our proofs. Let $a_1, \ldots, a_d$ be a basis of $\R^d$ contained in $\aaa$. Consider the dual basis $a_1^*, \ldots, a_d^*$:

\[
    \la a_i, a_j^* \ra = 
    \begin{cases}
        1,\, i = j \\
        0,\, i\neq j
    \end{cases}
\]
and observe that elements of $\bb$ have $0/1$ coordinates when expressed in this dual basis, or, in other words, $\bb$ is a subset of what we would call a cube: 
\[
    \bb \subseteq \left\{\sum_{i=1}^d \delta_i a_i^*, \text{ where } \delta_i \text{ range over } \{0, 1\}\right\}.
\]
Another observation is that projecting one family on the linear span of a subset of another preserves the binary scalar products property: if $\aaa' \subseteq \aaa$ and $\pi_{\aaa'}: \R^d \rightarrow \operatorname{span}(\aaa')$ is the orthogonal projection, then
\[
    \forall a \in \aaa',\, b \in \bb:\: \la a, \pi_{\aaa'}(b) \ra = \la a, b \ra \in \{0,1\}. 
\]

\noindent We will now introduce some notation and restate some claims proved in \cite{kupavskii22}. Proofs of those claims and inequalities are provided in Section \ref{secClaims} for completeness.

\noindent Since we are interested in bounding the product $|\aaa||\bb|$ from above, we will assume that $\aaa$ and $\bb$ are inclusion-wise maximal with respect to the property of having binary scalar products and, in paritcular, $\zero \in \aaa,\bb$. Let $b_d \in \bb \setminus \{\zero\}$ be a vector with the maximum value of $\operatorname{max}(\operatorname{dim}\aaa_0,\operatorname{dim}\aaa_1)$, where 
\[
    \aaa_i=\left\{a \in \aaa:\left\langle a, b_d\right\rangle=i\right\} \text{ for } i=0, 1.
\]
The choice of $b_d$ among the vectors that maximise $\operatorname{max}(\operatorname{dim}\aaa_0,\operatorname{dim}\aaa_1)$, in cases where it is important, will be specified at a later stage. %\textcolor{red}{Is it important?}
%\textcolor{olive}{Yes, for example in the final case of the main proof we choose $b_d$ to have the smallest size of $A_1$ }
We denote the orthogonal projection onto $U=b_d^\bot$ by $\pi: \mathbb{R}^d \rightarrow U$. We say that $X\subset \R^d$ does not contain opposite points if  $\{x,-x\}\subseteq X$ is only possible if $x=\zero$. Below, we state the claims and inequalities from \cite{kupavskii22}.

\begin{restatable}{claim}{claimassumptions}
    \label{cl1}
    We may translate $\aaa$ and replace some points $b\in\bb$ by the opposites $-b$ such that the following properties hold.
    \begin{itemize}
        \item[(i)] We (still) have $\aaa = \aaa_0 \cup \aaa_1$, where $\aaa_i = \{ a \in \aaa : \la a, b_d\ra = i\}$ for $i=0,1$ such that
            \begin{equation} \label{eqa0gea1}
                |\aaa_0| \ge |\aaa_1|.
            \end{equation}
        \item[(ii)] We  have
            \begin{equation}
                \label{eqscala0}
                \la a, b\ra \in \{0,1\} \text{ for each } a \in \aaa_0 \text{ and } b \in \bb.
            \end{equation}
        \item[(iii)] The set $\pi(\bb)$ does not contain opposite points.
    \end{itemize}
\end{restatable}

\begin{restatable}{claim}{claimpreimagespi}\label{cl2}
   Every point in $\pi(\bb)$ has at most two preimages in $\bb$.
\end{restatable}

We denote the linear span of $\aaa_0$ by $U_0$ and define the orthogonal projection $\tau: U \rightarrow U_0$. Let $\bb_* \subseteq \bb$ be the set of $b \in \bb$ for which $\pi(b)$ has exactly one preimage under projection onto $U$.

\begin{restatable}{inequality}{ineqbasic}\label{in0}
   $\left|\aaa\right|\left|\bb\right| \leq 2\left|\aaa_0\right||\pi(\bb)|+\left|\aaa_1\right|\left|\bb \backslash \bb_*\right|$.
\end{restatable}


\begin{restatable}{claim}{claimpreimagestau}\label{cl3}
    $\left|\pi(\bb)\right| \leq 2^{d-1-\operatorname{dim} U_0}\left|\tau(\pi(\bb))\right|$.
\end{restatable}

\begin{restatable}{claim}{claimrestbbconstant}\label{cl4}
    $\bb \backslash \bb_* $ can be partitioned as $ \bb_0 \sqcup \bb_1$, with $\bb_0, \bb_1$ satisfying
    \[
        \forall b\in\bb_i: \left|\left\{\langle a, b\rangle: a \in \aaa_i\right\}\right|=1\text{ for }i=0, 1.
    \]
\end{restatable}
%\textcolor{red}{I think that we need to more explicitly relate the second term in Inequality 1 and the two last terms in Inequality 2.}
%\textcolor{olive}{Fixed:\\}\textcolor{cyan}{
Due to Claim \ref{cl4} and \eqref{eqa0gea1}, the final term in Inequality \ref{in0} can be bounded as \[ |\aaa_1||\bb \backslash \bb_*| = |\aaa_1||\bb_0|+|\aaa_1||\bb_1| \leq |\aaa_0||\bb_0|+|\aaa_1||\bb_1|.\] Using this and applying Theorem~\ref{d_plus_one_two_d} to bound the first term in Inequality \ref{in0}, we obtain
\begin{restatable}{inequality}{ineqkey}\label{in1}
    $\left|\aaa\right| \cdot\left|\bb\right| \leq \left(\operatorname{dim} U_0+1\right) 2^d+\left|\aaa_0\right|\left|\bb_0\right|+\left|\aaa_1\right|\left|\bb_1\right|.$
\end{restatable}
\begin{restatable}{inequality}{ineqforclfive}\label{ineqForCl5}
    For $i = 0,1$ we have 
    \[
        |\aaa_i| \leq 2^{\operatorname{dim}(\aaa_i)},\;|\bb_i| \leq 2^{\operatorname{dim}(\operatorname{span}(\bb_i))} \text{, and } \operatorname{dim}(\aaa_i) + \operatorname{dim}\bigl(\operatorname{span}(\bb_i)\bigl) \leq d.
    \]
\end{restatable}

\begin{restatable}{claim}{claimaaaibbi}\label{cl5}
    For $i=0, 1$, we have $\left|\aaa_i\right|\left|\bb_i\right| \leq 2^d$.
\end{restatable}


\noindent Looking at the definition of $\bb_i$, we see that we can either assume  $\zero, b_d \in \bb_0$ or assume $\zero, b_d \in \bb_1$. Here and in what follows we assume that $\zero, b_d \in \bb_1$. Therefore, claim \ref{cl5} actually implies
\begin{equation}\label{stronger5}
    \left|\aaa_1\right|\left|\bb_1\right| \leq 2^d,\; \left|\aaa_0\right|\left(\left|\bb_0\right|+2\right)\leq 2^d.
\end{equation}

\noindent Inequality~\ref{in1} an Claim~\ref{cl5} are used in~\cite{kupavskii22} to prove Theorem~\ref{d_plus_one_two_d} as follows.\\ 
If $\operatorname{dim}U_0 \leq d-2$, 
\begin{equation}\label{final-old-step-smalldim}
    |\aaa| |\bb| \leq (\operatorname{dim}U_0 + 1)2^d + |\aaa_0||\bb_0|+|\aaa_1||\bb_1| \leq (d-1)2^d + 2^d + 2^d = (d+1)2^d
\end{equation}
If $\operatorname{dim}U_0 = d-1$, \eqref{ineqForCl5} and $\zero, b_d \in \bb_1$ implies $\bb_0 = \varnothing$, and we have
\begin{equation}\label{final-old-step-d-1}
    |\aaa| |\bb| \leq (\operatorname{dim}U_0 + 1)2^d + |\aaa_1||\bb_1| \leq d 2^d + 2^d = (d+1)2^d
\end{equation}

\section{Proofs of the main results}
\subsection{Uniqueness for Theorem~\ref{d_plus_one_two_d}}\label{sec31}

We start by proving a result that characterizes configurations that attain equality  in Theorem~\ref{d_plus_one_two_d}. This can be considered a warm-up proof, which is then used as a carcass for the proof of Theorem~\ref{d2d_plus_2d}. We heavily rely on the notation and claims introduced in the previous section.

\begin{theorem}\label{uniq}
    Let $\aaa,\bb \subseteq \R^d$ both linearly span $\R^d$ such that $\la a, b\ra \in \{0,1\}$ holds for all $a \in \aaa$, $b \in \bb$. Then we only have $|\aaa| \cdot |\bb| = (d+1) 2^d$ if one of the families has size $d+1$ and the other is affinely isomorphic to $\{0,1\}^d$.
\end{theorem}
\begin{proof}
     Without loss of generality, we assume $\left|\aaa\right| \geq \left|\bb\right|$. We  use induction on $d$, the statement being obvious in dimension $1$. Assuming that the statement holds for smaller dimensions, we prove it in dimension $d$. Consider two options for $\operatorname{dim}U_0$.
    \begin{enumerate}
        \item $\operatorname{dim}U_0 \leq d-2$. From Inequality \ref{in1} and \eqref{stronger5}, we get:
        \begin{equation*}
            \left|\aaa\right| \cdot\left|\bb\right| \leq 
            \left(d - 1\right) 2^d + 2\cdot2^d - 2\left|\aaa_0\right| \leq
            \left(d + 1\right) 2^d - \left|\aaa\right| < \left(d + 1\right) 2^d.
        \end{equation*}
        \item $\operatorname{dim}U_0 = d-1$. Note that since $\zero \in \aaa_0$, the definition of $\bb_0$ implies $\bb_0 \subset U_0^\bot$, and thus  we have $\bb_0 = \varnothing$ (recall that $\zero, b_d \in \bb_1$). We consider two subcases:
        \begin{enumerate}
            \item[a)]\label{uniqfirstcase} $\bb_*\neq\varnothing$. As we see from \eqref{final-old-step-d-1}, equality in Theorem~\ref{d_plus_one_two_d} can only be achieved when %\textcolor{red}{This needs to be explained more in the previous section: how do you combine the inequalities in order to obtain the bound. Otherwise, it's not clear here, I think.}\textcolor{cyan}{ Added \eqref{final-old-step-d-1} (and \eqref{final-old-step-smalldim}) for clarity} 
            Inequality \ref{in1} (and consequently Inequality \ref{in0}) are tight, which is only the case when  $\left|\aaa_0\right|\left|\pi(\bb)\right|=d 2^{d-1}$ (and $\left|\aaa_0\right|=\left|\aaa_1\right|$). By the induction hypothesis, the former is possible in one of two cases:
            \begin{enumerate}
                \item[i)] $\aaa_0$ is affinely isomorphic to $\{0,1\}^{d-1}$. Then, $\left|\aaa\right|=\left|\aaa_0\right|+\left|\aaa_1\right|=2^d$, which is only possible if $\aaa$ is affinely isomorphic to $\{0,1\}^d$, and then $\bb$ can only consist of a basis and the zero vector.
                \item[ii)] $\left|\aaa_0\right|=d$. Then, since $\left|\bb\right|\leq\left|\aaa\right|=2d$,  $\left|\aaa\right|\cdot\left|\bb\right|\leq4d^2$, which is less than $(d+1)2^d$ for $d\geq4$. For $d=3$, the inequality $\left|\bb\right|\cdot\left|\aaa\right|\leq32$ cannot yield equality since $\left|\aaa\right|=6$. Finally, if $d=2$ then $|\aaa|=4$, thus $\aaa$ is affinely isomorphic to a square  and $|\aaa|\cdot|\bb| = 3 \cdot 2^2$ only if $|\bb|=3=d+1$.
            \end{enumerate}
            \item [b)] $\bb_*=\varnothing$. Then, $\bb_1 = \bb$ and, consequently, $\operatorname{dim}(\operatorname{span}(\bb_1))=d$. In this case Inequality \ref{ineqForCl5} implies $|\aaa_1|=1$.
            Similarly to case a), Inequality \ref{in0} is only tight in one of the following cases:
            \begin{enumerate}
                \item[i)]$\left|\aaa_0\right|=d$. Then, $|\aaa|\cdot|\bb| \leq |\aaa|^2 \leq \left(d+1\right)^2<\left(d+1\right)2^d$.
                \item[ii)]$\left|\aaa_0\right|=2^{d-1}\text{, }\left|\pi(\bb)\right|=d$. Then, $\left|\aaa\right|\cdot\left|\bb\right|=2d\left(2^{d-1}+1\right)$, which is less than $(d+1)2^d$ for $d>2$. For $d=2$, we have $\left|\aaa\right|\cdot\left|\bb\right|\leq\left|\aaa\right|^2=9<3\cdot2^2$.
            \end{enumerate}
        \end{enumerate}
    \end{enumerate}
\end{proof}

\subsection{Proof of Theorem~\ref{d2d_plus_2d}}\label{sec32}
%\noindent We will improve the bound on $\left|\aaa\right|\cdot\left|\bb\right|$ for families that differ from the extremal example. To do this, 
For the proof, we  will need the following inequality. 

\begin{restatable}{inequality}{dfInequality}\label{in2}
    For an integer $2\leq f\leq d$, we have \[(d+f)(2^{d-1}+2^{d-f})\leq d2^d + 2d.\]
\end{restatable}

\noindent 
A short, but technical, proof of this inequality can be found in Appendix \ref{appendix}. 

For convenience, let us restate the theorem.
\mainth*
%\begin{proof}
    As in the proof of Theorem \ref{uniq}, we  use induction on $d$, and without loss of generality assume that $\left|\aaa\right|\geq\left|\bb\right|$. Note that we can also assume that $\aaa$ and $\bb$ are inclusion-wise maximal with respect to the property of having binary scalar products. For $d<3$, the bounds in Theorems~\ref{d2d_plus_2d} and~\ref{d_plus_one_two_d} coincide. Assuming validity for smaller dimensions, let us prove the statement for dimension $d$. We consider cases depending on the  value of $\operatorname{dim}U_0$.

    \begin{enumerate}
        \item\label{caseVerySmallU0} $\operatorname{dim}U_0 < d-2$. Then, from Inequality \ref{in1} and Claim \ref{cl5}, we have:
        \begin{equation}\label{strongWhenLessd-2}
            \left|\aaa\right|\cdot\left|\bb\right| \leq \left(\operatorname{dim} U_0 + 1\right) 2^d + 2^d + 2^d\leq d 2^d
        \end{equation}
        \item\label{caseSmallU0} $\operatorname{dim}U_0 = d-2$. Applying the induction hypothesis to the families $\tau(\pi(\bb))$ and $\aaa_0$, we have three cases:
        \begin{enumerate}
            \item[a)] $|\tau(\pi(\bb))|=d-1$. By maximality $\bb$ contained $\zero$, so $\tau(\pi(\bb))$ consists of zero and the basis of $U_0$. The maximality of $\aaa$ now implies that $\aaa_0$ is affinely isomorphic to $\{0,1\}^{d-2}$. From \eqref{stronger5}, it follows that $\left|\bb_0\right|\leq2$. 
            %If $b\in\bb_0$, then both elements \textcolor{red}{What do you mean by this exactly? please clarify}\textcolor{olive}{Fixed below.} of $\pi^{-1}(\pi(b))$ can be assumed to be in $\bb_0$, thus $\left|\bb_0\right|$ is even and we have two scenarios \textcolor{red}{probably, need to say that $|\pi(\bb)|=2d-2.$}\textcolor{olive}{hm $|\pi(\bb)|=2d-2$ is not neceseraly the case as we don't know the size of $\bb_1$}:
           % \textcolor{cyan}{\\
            For a given $b\in\bb_0$, there are two vectors that project onto $\pi(b)$ under $\pi$. Since they have identical scalar products with all the vectors in $\aaa_0$, and in our considerations below we work with $\aaa_0$ only, we can assume $|\bb_0|$ is even: if one vector belong to $\bb_0,$ then we can w.l.o.g. assume that the second one belongs to $\bb_0$ as well. We thus have two scenarios:
            \begin{enumerate}
                \item[i)] $\left|\bb_0\right|=0$. Then, from Inequality \ref{in0} and Claim \ref{cl5}, we obtain:
                \begin{equation*}
                    \left|\aaa\right|\cdot\left|\bb\right|\leq4\left(d-1\right)2^{d-2}+2^d=d 2^d
                \end{equation*}
                \item[ii)] $\left|\bb_0\right|=2$. Then $U_0^{\bot} \cap \bb$ consists of $\zero, b_d$ and two vectors from $\bb_0$. Let $\bb'$ be a subset of $\pi(\bb)$ containing all vectors $v$ such that $\tau(v)$ has two preimages in $\pi(\bb)$. Assume that $|\tau(\bb')|=k+1$ (and thus $|\bb'| = 2k+2$).
                Among these $k+1$ vectors, let $t_2$ be the number of those vectors with both preimages in $\pi(\bb_1)$, and let $t_1+1$ be the number of those with exactly one preimage in $\pi(\bb_1)$ (recall that $\bb_1$ includes zero). The remaining $k-t_1-t_2$ have both preimages in $\pi(\bb_*)$. Furthermore, let the vectors in $\tau(\pi(\bb))$ with a single preimage under $\tau$ consist of $q$ projections from $\pi(\bb_1)$ and $d-2-k-q$ projections from $\pi(\bb_*)$. Recall that by definition of $\bb_*$, $|\bb_*| = |\pi(\bb_*)|$, which consists of $t_1 + 2(k-t_1-t_2)$ vectors in $\bb'$ (out of $t_1 + 1$ vectors mentioned above, all but $\zero\in\R^d$ have the second preimage in $\pi(\bb_*)$) and $d-2-k-q$ vectors in $\pi(\bb)\setminus \bb'$. Therefore, 
                \[
                    |\bb_*| = t_1 + 2(k-t_1-t_2) + (d-2-k-q) = k - t_1 - 2t_2 + d - 2 - q.
                \]
                Next, from definition and Claim~\ref{cl2}, $|\bb_1|=2|\pi(\bb_1)|$. Besides $\zero$, $\pi(\bb_1)$ consists of $2t_2+t_1$ vectors from $\bb'$ and $q$ vectors from $\pi(\bb)\setminus \bb'$. This means that
                \[
                    |\bb_1| = 2|\pi(\bb_1)| = 2(t_1 + 2t_2 + q + 1) = 2 + 4t_2 + 2t_1 + 2q.
                \]
                Adding this up, we have:
                 
               % \st{We then have:} \textcolor{red}{this formula need an intermediate step. E.g., an explanation why the formulas for $|\bb_*|$ and $|\bb_1|$ are the way they are (explaining, which quantities of vectors we have in each group and to which group they contribute).}
                \begin{equation*}
                    \begin{split}
                        \left|\bb\right|
                        & = \left|\bb_*\right|+\left|\bb_0\right|+\left|\bb_1\right| \\
                        & = (k-t_1-2t_2+d-2-q)+2+(2+4t_2+2t_1+2q) \\ 
                        & = d + k + q + t_1 + 2t_2 + 2
                    \end{split}
                \end{equation*}
                First, consider the case when $t_2 > 0$. Then $\pi(\bb_1)$ contains two elements that differ by a vector orthogonal to $U_0$, thus $U_0^\bot\subset\operatorname{span}(\bb_1)$. Recall that $\tau(\pi(\bb))$ consists of zero and the basis of $U_0$, together with the previous observation this implies $\operatorname{dim}(\operatorname{span}(\bb_1)) = |\tau(\pi(\bb_1))|+1$. The family $\tau(\pi(\bb_1))$ consists of the zero vector,  $t_1 + t_2$ elements from $\tau(\bb')$ and $q$ elements from $\tau(\pi(\bb) \setminus \bb')$. We get that $\operatorname{dim}(\operatorname{span}(\bb_1)) = t_1+t_2+q+2$, and, by \eqref{ineqForCl5}, $|\aaa_1|\leq2^{d-t_1-t_2-q-2}$. Consequently,  $$|\aaa|= |\aaa_0| + |\aaa_1| \leq 2^{d-2} + 2^{d-2-t_1-t_2-q},$$ and we get the following chain of inequalities.
                %\st{which implies} \textcolor{red}{here, need to recall that $\aaa_0$ is a cube and so $\tau(\pi(\bb))$ is a basis. also, say sth like "the affine dimension is 2+number of points in $U_0$-1, and so...} :      
                \begin{align}
                    \left|\aaa\right|\cdot\left|\bb\right| 
                    & \leq \left(2^{d-2} + 2^{d-2-t_1-t_2-q}\right)\left(d+k+q+t_1+2t_2+2\right) \nonumber \\ 
                    & \leq \left(2^{d-2} + 2^{d-2-t_1-t_2-q}\right)\left(2d+t_1+2t_2\right) \label{mid1} \\
                    & \leq \left(2^{d-1} + 2^{d-1-t_1-t_2-q}\right)\left(d+t_1+t_2\right) \nonumber \\
                    & \leq \left(2^{d-1} + 2^{d-1-t_1-t_2}\right)\left(d+t_1+t_2\right) \label{mid2} \\
                    & \leq \left(2^{d-1} + 2^{d-1-t_1-t_2}\right)\left(d+t_1+t_2+1\right) \nonumber \\
                    & \leq d2^d + 2d. \label{mid3}             
                \end{align}
          %     \textcolor{red}{it still surprises me, how loose we are in this chain of inequalities. In particular, in the second to last - adding 1 to the second multiple.}\textcolor{olive}{Yes, there's quite a bit of slack here if $t_2\neq0$, but otherwise everything can be pretty tight - I think the exteremal example of the next "logical step" falls under this case} 
          Here, the second inequality follows from $k+q\leq d-2$, and the last one follows from Inequality \ref{in2}. If $t_2=0$, we get a slightly weaker bound:
                \begin{equation*}
                    \operatorname{dim}(\operatorname{span}(\bb_1)) \geq t_1+t_2+q+1 =t_1+q+1. \end{equation*}
                With the same reasoning this means that \eqref{mid2} becomes $\left(2^{d-1} + 2^{d-t_1}\right)\left(d+t_1\right)$, which is still less than \eqref{mid3} when $t_1\geq2$ due to Inequality \ref{in2}. Finally, when $t_2=0$ and $t_1=0, 1$, expression \eqref{mid1} yields a bound by $d2^d$ and $(2^{d-2}+2^{d-3})(2d+1)=d 2^d - \left(d-\frac{3}{2}\right)2^{d-2}\leq d2^d$, respectively.
            \end{enumerate}
            \item[b)] $|\aaa_0|=d-1$. Then:
            \begin{equation*}
                |\aaa|\cdot|\bb|\leq|\aaa|^2\leq4|\aaa_0|^2\leq4(d-1)^2\leq d2^d+2d, 
            \end{equation*}
            valid for any $d\ge 1.$
            \item[c)] Both $|\aaa_0|$ and $|\tau(\pi(\bb))|$ are at least $d$. By induction this implies 
            \[
            \left|\aaa_0\right|\cdot\left|\tau(\pi(\bb))\right|\leq\left(d-2\right)\left(2^{d-2}+2\right).
            \]
            Using Inequality \ref{in0}, claim \ref{cl3}, and \eqref{stronger5}, we have 
            \begin{equation*}
                \left|\aaa\right|\cdot\left|\bb\right| \leq 4\cdot\left(d-2\right)\left(2^{d-2}+2\right)+2 \cdot 2^d - 2\left|\aaa_0\right| = 2d(2^{d-1}+1)+2\left(3d-8-\left|\aaa_0\right|\right).
            \end{equation*}
            This completes the proof when $\left|\aaa_0\right|\geq3d-8$. Otherwise,
            \begin{equation*}
                \left|\aaa\right|\cdot\left|\bb\right| \leq \left|\aaa\right|^2 \leq 4\left|\aaa_0\right|^2 \leq 4\left(3d-9\right)^2,
            \end{equation*}
            which is less than $d2^d+2d$ for $d\geq3$. 
        \end{enumerate}
        \item $\operatorname{dim}U_0 = d-1$. Again, applying the induction hypothesis to $\pi(\bb)$ and $\aaa_0$, we have three cases (recall that from the assumption $\zero,b_d\in\bb_1$, we have $\bb_0=\varnothing$):
        \begin{enumerate}
            \item[a)] $|\pi(\bb)|=d$, that is, $\pi(\bb)$ consists of zero and the basis of $U_0$, which by maximality of $\aaa$ means that $\aaa_0$ is isomorphic to $\{0,1\}^{d-1}$.
            \begin{enumerate}
                \item[i)] $\operatorname{dim}\bb_1=1$. In this case, $\bb_1=\{\zero, b_d\}$ and so $\left|\bb\right|=d+1$. This contradicts the condition $|\bb|\ge d+2$ in the statement of the theorem.
                \item[ii)] $\operatorname{dim}\bb_1=k\geq2$. Then $\left|\bb_1\right|=2k$ and $\left|\aaa_1\right|\leq2^{d-k}$ by Inequality \ref{ineqForCl5}. Thus, we have
                \begin{equation*}
                    \left|\aaa\right|\cdot\left|\bb\right|\leq(2^{d-1}+2^{d-k})(d+k) \leq d2^d+2d
                \end{equation*}
                by Inequality \ref{in2}.
            \end{enumerate}
            \item[b)] $\left|\aaa_0\right|=d$. Then $|\aaa||\bb|\leq\left|\aaa\right|^2\leq4|\aaa_0|^2\leq4d^2$, which is not larger than $d2^d+2d$ for $d>3$. For $d=3$, $|\aaa|^2$ gives the desired bound when $|\aaa_1|\leq2$, and finally $|\aaa_1|=3$ would by Inequality \ref{ineqForCl5} imply 
            \[
                \operatorname{dim}\aaa_1=2 \Rightarrow |\bb_1|=2 \Rightarrow |\bb|\leq 5 \Rightarrow |\aaa|\cdot|\bb|\leq 3\cdot 2^3 + 2\cdot 3.
            \]
            \item[c)] Both $\left|\aaa_0\right|$ and $\left|\pi(\bb)\right|$ are at least $d+1$.
        \end{enumerate}
    \end{enumerate}
    The remainder of the proof deals with the case 3c). By the induction hypothesis, \begin{equation*}
            \left|\aaa_0\right|\cdot\left|\pi(\bb)\right|\leq\left(d-1\right)\left(2^{d-1}+2\right).
        \end{equation*}
        In the displayed chain below, we use Claim~\ref{cl2} in the first equality; in the fourth equality we use that $\bb_0 = \varnothing$, and thus $\bb\setminus \bb_*=\bb_1$; in the second inequality we use  Claim~\ref{cl5}.
        \begin{align}
            |\aaa|\cdot|\bb|& = (|\aaa_0|+|\aaa_1|)\cdot(2|\pi(\bb)|-|\bb_*|)\nonumber\\
            & = 2|\aaa_0||\pi(\bb)| + 2|\aaa_1||\pi(\bb)|- 2|\aaa_1||\bb_*| + |\aaa_1||\bb_*| - |\aaa_0||\bb_*|  \nonumber\\
            & = 2|\aaa_0||\pi(\bb)|  + |\aaa_1||\bb \setminus \bb_*| + |\aaa_1||\bb_*|- |\aaa_0||\bb_*| \nonumber\\
            & = 2|\aaa_0||\pi(\bb)|+|\aaa_1||\bb_1|-\left(|\aaa_0|-|\aaa_1|\right)|\bb_*| \label{general} \\
            & \leq 2\left(d-1\right)\left(2^{d-1}+2\right)+|\aaa_1||\bb_1|-\left(|\aaa_0|-|\aaa_1|\right)|\bb_*| \label{general_ind} \\
            & \leq 2\left(d-1\right)\left(2^{d-1}+2\right)+2^{d}-\left(|\aaa_0|-|\aaa_1|\right)|\bb_*| \nonumber \\
            & = d 2^{d} + 2d - \left(|\aaa_0|-|\aaa_1|\right)|\bb_*| + (2d-4). \label{2d_minus_4}
        \end{align}
        
        \noindent Thus, it suffices to show, for example, that 
        $\left(\left|\aaa_0\right|-\left|\aaa_1\right|\right)\left|\bb_*\right|\geq 2d-4$. \\
        \noindent First consider the case  $\operatorname{dim}\aaa_1=d-1$. Then $\bb_1=\{\zero, b_d\}$, and, using 
        \begin{equation*}
            |\aaa|\cdot|\bb|=|\aaa||\pi(\bb)|+|\aaa|\cdot\frac{1}{2}|\bb_1|\leq 2|\aaa_0||\pi(\bb)| + |\aaa| \leq d 2^d + 2d - 2^d + \left|\aaa\right|+(2d-4),
        \end{equation*}
    
        we obtain the desired inequality when $\left|\aaa\right|\leq 2^d-2d+4 $. Note that $\left|\aaa\right|> 2^d-2d+4$ is indeed impossible, as that would imply $\left|\aaa_0\right|> 2^{d-1}-d+2$ and
        \begin{equation*}
            \left|\aaa_0\right|\cdot\left|\pi(\bb)\right| > (2^{d-1}-d+2)\cdot(d+1)\geq(d-1)(2^{d-1}+2),
        \end{equation*}
        which contradicts the induction hypothesis. 
        
    In what follows, we assume that $\operatorname{dim}\aaa_1<d-1 $. Let us show that, due to this, we can also assume that $|\aaa_0|>|\aaa_1|$. To this end, suppose we had $|\aaa_0|=|\aaa_1|$ and recall how Claim~\ref{cl1} gave us opportunity to switch $\aaa_0$ and $\aaa_1$ places by translating $\aaa$ and replacing some points in $\bb$ by their opposites. Since $|\aaa_0|$ is not smaller then $|\aaa_1|$, this opportunity was not used, but since $|\aaa_0|=|\aaa_1|$, nothing stops us from employing this transform nevertheless. Since this has an effect of swapping $\aaa_0$ and $\aaa_1$, we reduce to a case where $\operatorname{dim}U_0<d-1$, for which the desired bound has been shown in cases \ref{caseVerySmallU0} and \ref{caseSmallU0}. Further, we assume $|\aaa_0| > |\aaa_1|$.

        Consider the orthogonal projection $\pi_{\bb_1}:\mathbb{R}^d\rightarrow\operatorname{span}(\bb_1)$. By the definition of $\aaa_1$, we have $\left|\pi_{\bb_1}(\aaa_1)\right|=1$. Let $k=\operatorname{dim(\operatorname{span}(\bb_1))}$. Since $\bb$ contains a basis of $\mathbb{R}^d$, we have
        \begin{equation}\label{A_dif_B_star}
            \left|\bb_*\right|\geq d-k ,\ 
            \left(\left|\aaa_0\right|-\left|\aaa_1\right|\right)\left|\bb_*\right| \geq d-k.
        \end{equation}
        We will now deal with possible values of $k$.
        \begin{enumerate}
            \item[i)] $k=1$, which means $\bb_1=\{0,\:b_d\}$. Since $\operatorname{dim}\aaa_1<d-1$, from Inequality~\ref{ineqForCl5} it follows that $\left|\aaa_1\right|\leq2^{d-2}$. Substituting this into \eqref{general_ind}, we obtain:
            \begin{equation*}
                |\aaa|\cdot|\bb|\leq d 2^d + 2d + (2d - 4 - 2^{d-1}) \leq d 2^d + 2d.
            \end{equation*}
            \item[ii)] $k=2$. From Inequality~\ref{ineqForCl5}, it follows that $\left|\bb_1\right|\leq4$, and $\left|\aaa_1\right|\leq2^{d-2}$. Due to \eqref{A_dif_B_star}, $\left|\bb_*\right|\geq d-2$, so if $\left|\aaa_0\right|-\left|\aaa_1\right|\geq 2$ then \eqref{2d_minus_4} yields the desired estimate. Similarly, \eqref{2d_minus_4} completes the proof if $\left|\aaa_0\right|-\left|\aaa_1\right|=1$ and $\left|\bb_*\right|\geq 2d-4$. Finally, if $\left|\aaa_0\right|-\left|\aaa_1\right|=1$ and $\left|\bb_*\right|<2d-4$, then:
            \begin{equation*}
                \left|\aaa\right|\cdot\left|\bb\right|=\left(2\left|\aaa_1\right|+1\right)\cdot\left(\left|\bb_*\right|+\left|\bb_1\right|\right)<\left(2^{d-1}+1\right)\cdot\left(2d-4+4\right)=d 2^d + 2d.
            \end{equation*}
            \item[iii)] $k=d$. Inequality \ref{ineqForCl5} implies that $\aaa_1$ consists of only one point. Hence, \eqref{general_ind} becomes
            \begin{equation*}
                \left|\aaa\right|\cdot\left|\bb\right|\leq 2\left(d-1\right)\left(2^{d-1}+2\right)+\left|\bb_1\right|-\left(\left|\aaa_0\right|-\left|\aaa_1\right|\right)\left|\bb_*\right| \leq 2\left(d-1\right)\left(2^{d-1}+2\right)+\left|\bb\right|,
            \end{equation*}
            which completes the proof when $|\bb| \leq 2^d-2d+4$. The opposite is indeed impossible, as it would contradict Theorem~\ref{d_plus_one_two_d}:
            \begin{equation*}
                \left|\aaa\right|\cdot\left|\bb\right|\geq\left|\bb\right|^2\geq\left(2^d-2d+4\right)^2>\left(d+1\right)2^d.
            \end{equation*}
        \end{enumerate}

        \noindent Before proceeding with the last case in the proof, we will note another structural property of $\aaa$, namely that when $k < d$ we can assume $|\pi_{\bb_1}(\aaa)|=k+1$. In other words, we claim we can assume $\pi_{\bb_1}(\aaa)$ consists of zero and the basis of $\operatorname{span}(\bb_1)$, with $\pi_{\bb_1}(\aaa_0)$ consisting of zero and the basis of $\operatorname{span}(\bb_1) \cap b_d^{\bot}$ (recall that $\pi_{\bb_1}(\aaa_1)$ contains a single point by definition of $\bb_1$). 
        Clearly $\aaa$ contains zero, and
        \[
            \operatorname{span}\bigl(\pi_{\bb_1}(\aaa)\bigr)=\operatorname{span}\bigl(\pi_{\bb_1}(\operatorname{span}(\aaa))\bigr)=\operatorname{span}(\bb_1),
        \]
        which means $\pi_{\bb_1}(\aaa)$ contains $\zero$ and a basis of an $k$-dimensional space, so $|\pi_{\bb_1}(\aaa)|\geq k+1$. Since by replacing some vectors in $\bb_1$ with their opposites (without affecting $|\bb_1|$) we ensure it has binary scalar products with $\aaa$, by Theorem \ref{d_plus_one_two_d} we have, if $\left|\pi_{\bb_1}(\aaa)\right| > k+1$,
        
        \begin{align}
            & \left|\bb_1\right|\cdot\left|\pi_{\bb_1}(\aaa)\right|\leq\left(k+1\right)2^k,\;\left|\pi_{\bb_1}(\aaa)\right|\geq k+2 \Rightarrow \left|\bb_1\right| \leq 2^k\left(1-\frac{1}{k+2}\right) \Rightarrow \nonumber \\
            & \left|\aaa_1\right|\left|\bb_1\right| \leq 2^d\left(1-\frac{1}{k+2}\right) \Rightarrow \left|\aaa\right|\cdot\left|\bb\right|\stackrel{\eqref{general_ind}}{\leq} d 2^d + 2d + (2d-4)- \frac{2^d}{k+2}-(d-k). \nonumber
        \end{align}
        This proves the required estimate, because for $d \geq 3$ and $k < d$
        \begin{equation*}
            d+k-4-\frac{2^d}{k+2}\leq2d-5-\frac{2^d}{d+1}=-\frac{1}{d+1}\left(2^d - (2d-5)(d+1)\right)\leq 0.
        \end{equation*}

        \noindent With the assumption of $|\pi_{\bb_1}(\aaa)|=k+1$, we proceed to the final subcase:
        
        \begin{enumerate}
            \item[iv)] $2 < k < d$. Note that, due to \eqref{A_dif_B_star}, $\bb_* \neq \varnothing$. Let's denote the elements of $\pi_{\bb_1}(\aaa)$ as $a_0 = 0, a_1, \ldots, a_k$, and their preimages in $\aaa$ as $\mathbb{A}_j=\pi_{\bb_1}^{-1}(a_j)\cap\aaa$. We'll choose the numbering such that $\mathbb{A}_1=\aaa_1$. Let $b_{11}, b_{12}, \ldots, b_{1k}$ be a basis of $\bb_1$ that is dual to $a_1, \ldots, a_k$. For example, according to our choice of numbering, $b_{11}=b_d$. Note that, due to $\bb$ being inclusion-wise maximal, all $b_{1j}$ must belong to $\bb_1$ (otherwise, they, along with $b_{1j}+b_d$ for $j>1$, could be added to $\bb$). If $\operatorname{dim}\aaa_1 < d-k$, we have $|\aaa_1||\bb_1|\leq 2^{d-1}$ and just like in part i) substitution into \eqref{general_ind} produces the desired estimate. Consequently, we can now assume that $\operatorname{dim}\aaa_1=d-k$.

            Our further plan is to write $\aaa$ in a particular basis to see that, due to $\operatorname{dim}\mathcal{A}_1=d-k$, any of the $b_{1j}$ could be initially chosen as $b_d$, and that a suitable choice of $b_d$ would lead to the desired bound.
            
            We will augment $\{b_{11}, \ldots, b_{1k}\}$ with elements from $\bb_*$ to form a basis for $\mathbb{R}^d$ and represent $\aaa$ in the dual basis. Then vectors of $\aaa$, arranged as column-vectors, form a matrix of the following form:
            
            \begin{equation*} 
            \renewcommand{\arraystretch}{0.9}
            \aaa=\begin{pNiceArray}{w{c}{2pt}w{c}{2pt}w{c}{2pt}w{c}{2pt}w{c}{2pt}|w{c}{2pt}w{c}{2pt}w{c}{2pt}w{c}{2pt}w{c}{2pt}|w{c}{2pt}w{c}{2pt}w{c}{2pt}w{c}{2pt}w{c}{2pt}|w{c}{2pt}w{c}{2pt}w{c}{2pt}w{c}{2pt}w{c}{2pt}|w{c}{2pt}w{c}{2pt}w{c}{2pt}w{c}{2pt}w{c}{2pt}}[first-row,last-col]
                \Block{1-5}{\mathbb{A}_0}\phantom{0} & & & & &
                \Block{1-5}{\mathbb{A}_1} & & & & &
                \Block{1-5}{\mathbb{A}_2} & & & & &
                \Block{1-5}{\cdots} & & & & &
                \Block{1-5}{\mathbb{A}_k} & & & & &\\
                
                \Block{5-5}<\LARGE>{\bm{0}} & & & & &
                1 & 1 & \cdots & 1 & 1 & 
                0 & 0 & \cdots & 0 & 0 &
                \Block{10-5}<\Large>{\cdots} & & & & &
                \Block{4-5}<\Large>{\bm{0}} & & & & &\\
                
                  &  &  &  &  &
                \Block{4-5}<\Large>{\bm{0}} & & & & & 
                1 & 1 & \cdots & 1 & 1 &
                  &  &  &  &  & 
                  &  &  &  &  &\\

                  &  &  &  &  &
                  &  &  &  &  &
                \Block{3-5}<\Large>{\bm{0}} &  &  &  &  &
                  &  &  &  &  &
                  &  &  &  &  &  \ \:k\\
                  
                &&&&&&&&&&&&&&&&&&&&&&&&\\
                &&&&&&&&&&&&&&&&&&&& 1 & 1 & \cdots & 1 & 1 &\\ \hdottedline

                \phantom{0}&&&&&\Block[fill=[RGB]{235,235,235},respect-arraystretch]{5-5}{}&&&&&\phantom{0}&&&&&\phantom{0}&&&&&\phantom{0}&&&& &\\
                &\phantom{0}&&&&&\phantom{0}&&&&&\phantom{0}&&&&&\phantom{0}&&&&&\phantom{0}&&& &\\
                &&\phantom{0}&&&&&\phantom{0}&&&&&\phantom{0}&&&&&\phantom{0}&&&&&\phantom{0}&& &\ \:d-k\\
                &&&\phantom{0}&&&&&\phantom{0}&&&&&\phantom{0}&&&&&\phantom{0}&&&&&\phantom{0}& &\\
                &&&&\phantom{0}&&&&&\phantom{0}&&&&&\phantom{0}&&&&&\phantom{0}&&&&&\phantom{0} &\\
                
                \CodeAfter
                    \line[radius=0.7pt]{7-2}{9-4}
                    \line[radius=0.7pt]{7-7}{9-9}
                    \line[radius=0.7pt]{7-12}{9-14}
                    \line[radius=0.7pt]{7-22}{9-24}
                    \SubMatrix{.}{1-1}{5-25}{]}[right-xshift=7pt]
                    \SubMatrix{.}{6-1}{10-25}{]}[right-xshift=7pt]
                    %\UnderBrace[left-shorten, yshift=5pt]{6-6}{10-10}{\substack{\operatorname{dim}=d-k\\\textcolor{cyan}{\operatorname{rank}=d-k+1}}}
                    \UnderBrace[left-shorten, yshift=5pt]{6-6}{10-10}{\operatorname{dim}=d-k} % I like it more without the rank 
                \end{pNiceArray}
            \end{equation*}

            \bigskip\bigskip\bigskip  %% TODO replace by a couple of phantom lines wrapped in "samepage"
            
            The affine dimenion of the highlighted block coincides with the affine dimension of $\mathbb{A}_1=\aaa_1$, which is $d-k$. There is therefore a basis of a $\R^d$ that consists of $d-k+1$ vectors from $\mathcal{A}_1$ and one vector from each other $\mathbb{A}_l$, $l>1$. Since $\zero \in \mathbb{A}_0$, this means $\operatorname{dim}(\aaa\setminus \mathbb{A}_j)=d-1$ for all $j>1$. We thus have
            \begin{equation*}
                \forall j>1\!:\;\operatorname{dim}(\aaa\cap b_{1j}^\bot) = \operatorname{dim}(\aaa\setminus\mathbb{A}_j) = d-1,
            \end{equation*}  
            which means that, indeed, any of the $b_{1j}$ could be set as $b_d$ from the start. Choose $b_{1j}$ with the smallest possible size of $\mathbb{A}_j$, and repeat all the same reasoning with it as $b_d$. Note that in this case, $\left|\aaa\setminus\mathbb{A}_j\right|>\left|\mathbb{A}_j\right|$, so there will be no need for translation of $\aaa$ that swaps $\aaa_0$ and $\aaa_1$ in Claim~\ref{cl1}. After this reassignment of $b_d$ and appropriate relabeling of families $\mathbb{A}_l$, we may assume that $|\mathbb{A}_j|$, among positive $j$, is minimised by $j=1$.
            \begin{align}
                \forall j>1\!:\:|\mathbb{A}_1| \leq |\mathbb{A}_j|\; \Longrightarrow & \nonumber\\   |\aaa_0|-|\aaa_1| = \biggl(|\mathbb{A}_0|+\sum_{j>1}|\mathbb{A}_j|\biggr) - |\mathbb{A}_1| > \sum_{j>1}|\mathbb{A}_j| - |\mathbb{A}_1| & \geq(k-2)|\aaa_1| \geq |\aaa_1|.\label{dif_A1_bound}
            \end{align} 
            If $\left|\aaa_0\right|-\left|\aaa_1\right|\geq2d-4$, non-emptiness of $\bb_*$ and \eqref{2d_minus_4} imply the desired estimate. Otherwise
            \begin{align*}
                |\aaa_0|-|\aaa_1| \leq 2d-5 & \;\xRightarrow{\eqref{dif_A1_bound}}\; |\aaa_1| \leq 2d - 6\;\Longrightarrow |\aaa| = (|\aaa_0|-|\aaa_1|) + 2|\aaa_1| \leq 6d-17, \\
                &|\aaa|\cdot|\bb| \leq |\aaa|^2 \leq (6d-17)^2 < d 2^d + 2d,
            \end{align*}
            concluding the proof.
        \end{enumerate}
%\end{proof}

