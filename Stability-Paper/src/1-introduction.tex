\section{Introduction}

A polytope $P$ is {\it 2-level} if for every facet-defining hyperplane $H$ there is a parallel hyperplane $H'$ such that $H \cup H'$ contains all vertices of $P$. Basic examples of 2-level polytopes are simplices, hypercubes and cross-polytopes, but they also generalize a variety of interesting polytopes such as Birkhoff, Hanner, and Hansen polytopes, order polytopes and chain polytopes of posets, stable matching polytopes, and stable set polytopes of perfect graphs~\cite{aprile18}. Combinatorial structure of two-level polytopes has also been studied in~\cite{fiorini16}, and enumeration of such polytopes in~\cite{bohn18} led to a beautiful conjecture about their vertex and facet count, which was proven in~\cite{kupavskii22}:
\begin{restatable}{theorem}{twoLevelOld}
    \label{two_level_old_bound}
    If $P$ is a $d$-dimensional 2-level polytope, it's number of vertices $f_0(P)$ and facets $f_{d-1}(P)$ satisfy
    \[
        f_0(P) \cdot f_{d-1}(P) \leq d 2^{d+1}.
    \]
\end{restatable}
\noindent This bound is tight, as is witnessed by polytopes that are affinely isomorphic to the cube or the cross-polytope. Authors of~\cite{bohn18} conjectured that those are the only instances where equality is attained (see \cite{aprile18}). In this paper, we prove this in a strong sense:

\begin{restatable}{theorem}{twoLevelNew}
    \label{two_level_new_bound}
    Fix $d>1$. Let $P$ be a $d$-dimensional $2$-level polytope that is not affinely isomorphic to the cube or the cross-polytope. Then 
    \[
        f_0(P) \cdot f_{d-1}(P) \leq \left(d-1\right) 2^{d+1} + 8\left(d-1\right).
    \]
\end{restatable}
 \noindent The following two examples demonstrate tightness of the bound in Theorem \ref{two_level_new_bound}.
\begin{example}[Suspension of a cube]\label{CubeSuspension}
    Let $\{e_i\}$ be the standard basis of $\mathbb{R}^d$,
    \begin{equation*}
        P=\operatorname{Conv}\Biggl(\left\{\sum_{i=1}^{d-1}\varepsilon_i e_i\right\}\cup\left\{e_d, -e_d\right\}\Biggr)\text{, where }\varepsilon_i\text{ range over }\left\{-1, 1\right\}.
    \end{equation*}
 Here $f_0(P) = 2 + 2^{d-1}$ and $f_{d-1}(P) = 4(d-1)$.
\end{example}
\begin{example}[Cross-polytope $\times$ segment]\label{OctahedronCrossSegment}
    Let $\{e_i\}$ be the standard basis of $\mathbb{R}^d$,
    \begin{equation*}
        P=\operatorname{Conv}\bigl(\left\{\eps_i e_i + \eps_d e_d\right\}_{i\leq d-1}\bigr)\text{, where }\eps_i\text{ range over }\left\{-1, 1\right\}\text{ for $i\in[d]$}.
    \end{equation*}
      This is (up to coordinate scaling) the dual of the polytope in the previous example and, in particular,  $f_0(P) = 4(d-1)$ and $f_{d-1}(P) = 2 + 2^{d-1}$. 
\end{example}


\noindent As in the paper \cite{kupavskii22}, the main intermediate result that is of independent interest concerns families of vectors with binary scalar products.

\begin{restatable}[]{theorem}{mainth}\label{d2d_plus_2d}
     Let $\aaa,\bb \subseteq \R^d$ be families of vectors that both linearly span $\R^d$. Suppose that $\la a, b\ra \in \{0,1\}$ holds for all $a \in \aaa$, $b \in \bb$. Furthermore, suppose that $|\aaa|,|\bb|\ge d+2$. Then $\left|\mathcal{A}\right| \cdot\left|\mathcal{B}\right| \leq d 2^d + 2d$.
\end{restatable}

\noindent In other words, this theorem is a tight stability result for the bound in the following theorem, which was the main result of~\cite{kupavskii22}.

\begin{restatable}{theorem}{oldMainTh}
    \label{d_plus_one_two_d}
    Let $\aaa,\bb \subseteq \R^d$ both linearly span $\R^d$ such that $\la a, b\ra \in \{0,1\}$ holds for all $a \in \aaa$, $b \in \bb$.
    Then we have $|\aaa| \cdot |\bb| \le (d+1) 2^d$.
\end{restatable}

\noindent Two examples that demonstrate tightness of the bound in Theorem \ref{d2d_plus_2d} are

\begin{example}\label{cubeOctop}
    Let $\{e_i\}$ be the standard basis of $\R^d$, 
    \begin{equation*}
        \aaa=\left\{\sum_{i=2}^{d}\delta_i e_i\right\}\cup\left\{e_1\right\}\text{, }\bb=\left\{\delta_1 e_1 + e_j\right\} \cup \left\{e_1, 0\right\}\text{, where }\delta_i\text{ range over }\left\{0, 1\right\}\text{ and }j\text{ over }[2,d].
    \end{equation*}
    Here $\left|\mathcal{A}\right|=2^{d-1}+1$ and $\left|\mathcal{B}\right|=2d$.
\end{example}

\begin{example}\label{crosspoly}
    Let $\{e_i\}$ be the standard basis of $\mathbb{R}^d$,
    \begin{equation*}
        \aaa=\left\{e_d+\sum_{i=1}^{d-1}\varepsilon_i e_i\right\}\cup\left\{0\right\}\text{, }\bb=\left\{\frac{1}{2}\left(e_d+\varepsilon_i e_i\right)\right\}\text{, where }\varepsilon_i\text{ range over }\left\{-1, 1\right\}\text{ and $i$ over }[d].
    \end{equation*}
    As in Example \ref{cubeOctop}, $\left|\aaa\right|=2^{d-1}+1$ and $\left|\bb\right|=2d$.
\end{example}

The proof of Theorem~\ref{d2d_plus_2d} builds on the proof of Theorem~\ref{d_plus_one_two_d}, thus, in the next section we present the necessary claims and inequalities from \cite{kupavskii22}. In Section~\ref{sec31} we prove a baby variant of Theorem~\ref{d2d_plus_2d}, that gives uniqueness of the extremal example for Theorem~\ref{d_plus_one_two_d}. The structure of this proof is then reused in Section~\ref{sec32}, where we prove Theorem~\ref{d2d_plus_2d}. In Section~\ref{sec2level} we prove our main result, Theorem~\ref{two_level_new_bound}. 

The proofs of our main results build on the proofs from \cite{kupavskii22}, but require several new ingredients, both combinatorial and, most importantly, geometric. Unfortunately, there is quite a bit of case analysis involved, one reason being that there are actually many different configurations that are close to the bound in Theorem~\ref{d2d_plus_2d}. Filtering all of them out requires different considerations. Geometrically, the most interesting cases are: 3c in the proof of Theorem~\ref{d2d_plus_2d}, where $\aaa$ and $\bb$ switch roles, and we have to study a projection onto a certain subspace formed by vectors of $\bb$, followed by adding a twist on the choice of the vector $b_d$, along which we project in order to use induction; the last case in the proof of Theorem~\ref{two_level_new_bound}, in which we reveal the exact geometric structure of $\aaa$ by reducing the problem to a simple question about families of subsets of $[d]$ with small pairwise differences.  

\paragraph{Outline}
The next section lays out the proof of our main tool. In Section~\ref{sec2level} we provide the proof of Theorem~\ref{two_level_new_bound} and Section~\ref{secClaims} contains proofs of claims from \cite{kupavskii22} that we use. Short but technical proofs of some statements used in the main sections are provided in Appendix~\ref{appendix}, as well as a conjecture that generalises Theorem~\ref{d2d_plus_2d}.