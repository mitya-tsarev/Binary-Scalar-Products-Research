\section{Introduction}

A polytope $P$ is said to be 2-level if for every facet-defining hyperplane $H$ there is a parallel hyperplane $H'$ such that $H \cup H'$ contains all vertices of $P$. Basic examples of 2-level polytopes are simplices, hypercubes and cross-polytopes, but they also generalize a variety of interesting polytopes such as Birkhoff, Hanner, and Hansen polytopes, order polytopes and chain polytopes of posets, stable matching polytopes, and stable set polytopes of perfect graphs~\cite{aprile18}. Combinatorial structure of two-level polytopes has also been studied in~\cite{fiorini16}, and enumeration of such polytopes in~\cite{bohn18} led to a beautiful conjecture about their vertex and facet count, which was proven in~\cite{kupavskii22}:
\begin{restatable}{theorem}{twoLevelOld}
    \label{two_level_old_bound}
    If $P$ is a $d$-dimensional 2-level polytope, it's number of vertices $f_0(P)$ and facets $f_{d-1}(P)$ satisfy
    \[
        f_0(P) \cdot f_{d-1}(P) \leq d 2^{d+1}.
    \]
\end{restatable}
\noindent This bound is tight by considering $P$ that is affinely isomorphic to the cube or the cross-polytope. Authors of~\cite{bohn18} conjectured that those are the only instances where equality is attained (personal communication). In this paper, we prove this in a strong sense:

\begin{restatable}{theorem}{twoLevelNew}
    \label{two_level_new_bound}
    For $d>1$ let $P$ be a $d$-dimensional $2$-level polytope that is not affinely isomorphic to the cube or the cross-polytope. Then 
    \[
        f_0(P) \cdot f_{d-1}(P) \leq \left(d-1\right) 2^{d+1} + 8\left(d-1\right).
    \]
\end{restatable}

\noindent Our main tool is going to be a stronger theorem regarding so-called families of vectors with binary scalar products:

\begin{restatable}[]{theorem}{mainth}\label{d2d_plus_2d}
     Let $\aaa,\bb \subseteq \R^d$ both linearly span $\R^d$ such that $\la a, b\ra \in \{0,1\}$ holds for all $a \in \aaa$, $b \in \bb$. Furthermore, suppose $\aaa$ and $\bb$ both have the size of at least $d+2$. Then $\left|\mathcal{A}\right| \cdot\left|\mathcal{B}\right| \leq d 2^d + 2d$.
\end{restatable}

\noindent In other words, our main tool is the stability of the bound in Theorem \ref{d_plus_one_two_d}, which was the main result of~\cite{kupavskii22}:

\begin{restatable}{theorem}{oldMainTh}
    \label{d_plus_one_two_d}
    Let $\aaa,\bb \subseteq \R^d$ both linearly span $\R^d$ such that $\la a, b\ra \in \{0,1\}$ holds for all $a \in \aaa$, $b \in \bb$.
    Then we have $|\aaa| \cdot |\bb| \le (d+1) 2^d$.
\end{restatable}

\paragraph{Notation}
In what follows, We will often treat vectors in $\R^d$ as points in an affine space, with $\operatorname{dim}$ always referring to the affine dimension while $\operatorname{span}$ referrs to linear span. The set of integers  from $1$ to $n$ is denoted $[n]$. 

\paragraph{Outline}
The next section lays out the proof of our main tool. In Section~\ref{sec2level} we provide the proof of Theorem~\ref{two_level_new_bound} and Section~\ref{secClaims} contains proofs of claims from \cite{kupavskii22} that we use. Short but technical proofs of some statements used in the main sections are provided in Appendix~\ref{appendix}, as well as a conjecture that generalises Theorem~\ref{d2d_plus_2d}.