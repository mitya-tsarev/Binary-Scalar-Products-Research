\section{Introduction}

A polytope $P$ is {\it 2-level} if for every facet-defining hyperplane $H$ there is a parallel hyperplane $H'$ such that $H \cup H'$ contains all vertices of $P$. Basic examples of 2-level polytopes are simplices, hypercubes and cross-polytopes, but they also generalize a variety of interesting polytopes such as Birkhoff, Hanner, and Hansen polytopes, order polytopes and chain polytopes of posets, stable matching polytopes, and stable set polytopes of perfect graphs~\cite{aprile18}. Combinatorial structure of two-level polytopes has also been studied in~\cite{fiorini16}, and enumeration of such polytopes in~\cite{bohn18} led to a beautiful conjecture about their vertex and facet count, which was proven in~\cite{kupavskii22}:
\begin{restatable}{theorem}{twoLevelOld}
    \label{two_level_old_bound}
    If $P$ is a $d$-dimensional 2-level polytope, its number of vertices $f_0(P)$ and facets $f_{d-1}(P)$ satisfy
    \[
        f_0(P) \cdot f_{d-1}(P) \leq d 2^{d+1}.
    \]
\end{restatable}
\noindent This bound is tight, as is witnessed by polytopes that are affinely isomorphic to the cube or the cross-polytope. Authors of~\cite{bohn18} conjectured that those are the only instances where equality is attained (see \cite{aprile18}). In this paper, we prove this in a strong sense:

\begin{restatable}{theorem}{twoLevelNew}
    \label{two_level_new_bound}
    Fix $d>1$. Let $P$ be a $d$-dimensional $2$-level polytope that is not affinely isomorphic to the cube or the cross-polytope. Then 
    \[
        f_0(P) \cdot f_{d-1}(P) \leq \left(d-1\right) 2^{d+1} + 8\left(d-1\right).
    \]
\end{restatable}
 \noindent The following two examples demonstrate tightness of the bound in Theorem \ref{two_level_new_bound}.
\begin{example}[Suspension of a cube]\label{CubeSuspension}
    Let $\{e_i\}$ be the standard basis of $\mathbb{R}^d$,
    \begin{equation*}
        P=\operatorname{Conv}\Biggl(\left\{\sum_{i=1}^{d-1}\varepsilon_i e_i: \varepsilon_i\in\{-1, 1\} \text{ for }i\in [d-1]\right\}\cup\left\{e_d, -e_d\right\}\Biggr).
    \end{equation*}
 Here $f_0(P) = 2 + 2^{d-1}$ and $f_{d-1}(P) = 4(d-1)$.
\end{example}
\begin{example}[Cross-polytope $\times$ segment]\label{OctahedronCrossSegment}
    Let $\{e_i\}$ be the standard basis of $\mathbb{R}^d$,
    \begin{equation*}
        P=\operatorname{Conv}\bigl(\left\{ \eps_i e_i+\eps_d e_d: i\in [d-1], \eps_i,\eps_d\in \{-1,1\}\right\}\bigr).
    \end{equation*}
      This is (up to coordinate scaling) the dual of the polytope in the previous example and, in particular,  $f_0(P) = 4(d-1)$ and $f_{d-1}(P) = 2 + 2^{d-1}$. 
\end{example}


\noindent As in the paper \cite{kupavskii22}, the main intermediate result that is of independent interest concerns families of vectors with binary scalar products.

\begin{restatable}[]{theorem}{mainth}\label{d2d_plus_2d}
     Let $\aaa,\bb \subseteq \R^d$ be families of vectors that both linearly span $\R^d$. Suppose that $\la a, b\ra \in \{0,1\}$ holds for all $a \in \aaa$, $b \in \bb$. Furthermore, suppose that $|\aaa|,|\bb|\ge d+2$. Then \begin{equation}\label{eqmainvec}\left|\mathcal{A}\right| \cdot\left|\mathcal{B}\right| \leq d 2^d + 2d.\end{equation}
\end{restatable}

\noindent This theorem is in fact a tight stability result for the following theorem, which was the main result of~\cite{kupavskii22}.

\begin{restatable}{theorem}{oldMainTh}
    \label{d_plus_one_two_d}
    Let $\aaa,\bb \subseteq \R^d$ both linearly span $\R^d$ such that $\la a, b\ra \in \{0,1\}$ holds for all $a \in \aaa$, $b \in \bb$.
    Then we have $|\aaa| \cdot |\bb| \le (d+1) 2^d$.
\end{restatable}

\noindent We give two examples that demonstrate tightness of the bound in Theorem \ref{d2d_plus_2d}.

\begin{example}\label{cubeOctop}
    Let $\{e_i\}$ be the standard basis of $\R^d$, 
    \begin{align*}
        \aaa=&\left\{\sum_{i=2}^{d}\delta_i e_i: \delta_i\in \{0,1\} \text{ for all }i\in [2,d]\right\}\cup\left\{e_1\right\},\\ 
        \bb=&\big\{\delta_1 e_1 + e_j: j\in [2,d] \text{ and }\delta_1\in \{0,1\}\big\} \cup \left\{e_1, 0\right\}.
    \end{align*}
    Here $\left|\mathcal{A}\right|=2^{d-1}+1$ and $\left|\mathcal{B}\right|=2d$.
\end{example}
The example above has both subsets within the binary cube, thus it can be interpreted as two families of sets $\aaa$ and $\bb$ such that $\forall A\in \aaa,\,B\in\bb$ we have $|\aaa\cap\bb|\in\{0,1\}$. This does not hold for the example below.
\begin{example}\label{crosspoly}
    Let $\{e_i\}$ be the standard basis of $\mathbb{R}^d$,
    \begin{align*}
        \aaa=&\left\{e_d+\sum_{i=1}^{d-1}\varepsilon_i e_i: \eps_i \in \{-1,1\} \text{ for all }i\in [d-1]\right\}\cup\left\{0\right\},\\
        \bb=&\left\{\frac{1}{2}\left(e_d+\varepsilon_i e_i\right): i\in [d], \eps_i\in \{-1,1\}\right\}.
    \end{align*}
    As in Example \ref{cubeOctop}, $\left|\aaa\right|=2^{d-1}+1$ and $\left|\bb\right|=2d$.
\end{example}

\paragraph{Outline}
The proof of Theorem~\ref{d2d_plus_2d} builds on the proof of Theorem~\ref{d_plus_one_two_d}, thus, in the next section we present the necessary claims and inequalities from \cite{kupavskii22}. In Section~\ref{sec31} we prove a baby variant of Theorem~\ref{d2d_plus_2d}, that gives uniqueness of the extremal example for Theorem~\ref{d_plus_one_two_d}. The structure of this proof is then reused in Section~\ref{sec32}, where we prove Theorem~\ref{d2d_plus_2d}. In Section~\ref{sec2level} we prove our main result, Theorem~\ref{two_level_new_bound}.
Proofs of claims from \cite{kupavskii22} are provided in Appendix~\ref{appendix} to make this paper is self-contained.

\paragraph{Discussion of the proofs} The proofs of our main results build on the proofs from \cite{kupavskii22}, but require several new ingredients, both combinatorial and, most importantly, geometric. The general idea is to project our families onto a certain subspace and make use of induction on the dimension of the ambient space. Unfortunately, there is quite a bit of case analysis involved, one reason being that there are actually many different configurations that are close to the bound in Theorem~\ref{d2d_plus_2d}. This is witnessed by some of our computer enumeration results below and by explicit constructions that are similar in spirit to Examples~\ref{cubeOctop} and~\ref{crosspoly}, for instance, Example~\ref{generalCubeOctop} discussed below. Filtering all of them out requires different considerations. Another reason is that Theorem~\ref{d2d_plus_2d} has a condition on the sizes of $\aaa$ and $\bb$, and thus before invoking induction hypothesis we must deal with the cases where one of the projected families is small. Geometrically, the most interesting cases are: 3c in the proof of Theorem~\ref{d2d_plus_2d}, where $\aaa$ and $\bb$ switch roles, and we have to study a projection onto a certain subspace formed by vectors of $\bb$, followed by adding a twist on the choice of the vector $b_d$, along which we project in order to use induction; the last case in the proof of Theorem~\ref{two_level_new_bound}, in which we reveal the exact geometric structure of $\aaa$ by reducing the problem to a simple question about families of subsets of $[d-1]$ with small pairwise differences.  


By utilising some observations made at the beginning of Section~\ref{secStability} we were able to enumerate all families with binary scalar products (up to linear isomorphism of individual sets) in dimensions $d \leq 5$. The maximal (with respect to the product order on $\mathbb{N}\times\mathbb{N}$) pairs $(|\aaa|,|\bb|)$ for \mbox{$4$-dimensional} families are $(5, 16),\,(6, 12),\,(7, 10),\,(8, 9),\,(9, 8),\,(10, 7),\,(12, 6)$ and $(16, 5)$, which, together with examples above, demonstrates that the bound of $d 2^d + 2d$ might be achieved on sets of different structures (pairs $(6,12), (8,9), (9,8)$ and $(12,6)$ above). Figure~\ref{posSizesPlot} depicts all possible sizes of families in $\R^5$, Figure~\ref{pos-Min-Prod-Plot} shows the same data plotted with $|\aaa|\cdot|\bb|$ against $\operatorname{min}(|\aaa|,|\bb|)$ for clarity. 

% TODO Refine figure placement/sizes after text is settled
\begin{figure}[!h]
\centering
\begin{tikzpicture}
\begin{axis}[
    xlabel={$|\aaa|$},
    ylabel={$|\bb|$},
    axis lines=left,
    xmin=3, xmax=34,
    ymin=3, ymax=34, 
    legend pos=north east,
    legend style={font=\scriptsize},
    width=0.5\textwidth,
    height=0.5\textwidth,
    label style={font=\small}
]
    
\addplot[domain=5:34, samples=100, color=black, dashed] {192/x};
\addlegendentry{$x\cdot y = (5+1)2^5$}

\addplot[only marks, color=black, mark=*, mark options={fill=black},mark size=1pt] 
    table {all5.dat};

\end{axis}
\end{tikzpicture}
\caption{Possible sizes of families $\aaa$, $\bb$ that span $\R^5$ and have binary scalar products.}
\label{posSizesPlot}
\end{figure}

% TODO Refine figure placement/sizes after text is settled
\begin{figure}[!h]
\centering
\begin{tikzpicture}
\begin{axis}[
    xlabel={$\operatorname{min}(|\aaa|,|\bb|)$},
    ylabel={$|\aaa|\cdot|\bb|$},
    axis lines=left,
    xmin=4, xmax=14,
    ymin=0, ymax=215, 
    legend pos=south east,
    legend style={font=\scriptsize},
    width=0.65\textwidth,
    height=0.5\textwidth,
    label style={font=\small}
]

\addplot[domain=3:13, samples=100, color=black, dashed] {192};
\addlegendentry{$y = (5+1)2^5$}

\addplot[domain=3:13, samples=100, color=black, line width=0.8pt, dotted] {170};
\addlegendentry{$y = 5\cdot2^5+2\cdot5$}

\addplot[only marks, color=black, mark=*, mark options={fill=black},mark size=1pt] 
    table {all5-min-vs-prod.dat};

\end{axis}
\end{tikzpicture}
\caption{$\operatorname{min}(|\aaa|, |\bb|)$ and $|\aaa||\bb|$ for families that span $\R^5$ and have binary scalar products.}
\label{pos-Min-Prod-Plot}
\end{figure}


The proof of Theorem~\ref{two_level_new_bound} makes use of Theorem~\ref{d2d_plus_2d} and a quick observation that $P$ is affinely isomorphic to the cube if it has too few facets to apply Theorem~\ref{d2d_plus_2d}. However, the case where $P$ has few vertices cannot be easily reduced to the case with few facets, as the set of possible pairs $\left(f_0(P), f_{d-1}(P)\right)$ for $2$-level $P$ is not symmetric. For example, a triangular prism in $\R^3$ is $2$-level, but there is no $3$-dimensional $2$-level $P$ satisfying $f_0(P)=5$ and $f_2(P)=6$.

\noindent We finish with  a conjecture that generalises Theorems~\ref{d_plus_one_two_d} and~\ref{d2d_plus_2d}.


\begin{conjecture}\label{generalisation}
    Let $\aaa,\bb \subseteq \R^d$ be families of vectors that both linearly span $\R^d$. Suppose that $\la a, b\ra \in \{0,1\}$ holds for all $a \in \aaa$, $b \in \bb$. Furthermore, suppose that $|\aaa|$ and $|\bb|$ are both strictly larger than $2^{k-1}(d-k+2)$ for some $k\in [0,d]$. Then $\left|\mathcal{A}\right| \cdot\left|\mathcal{B}\right| \leq (2^{d-k}+k)2^k(d-k+1)$.
\end{conjecture}
The motivating example for this conjecture is the following generalisation of Example~\ref{cubeOctop}:
\begin{example}\label{generalCubeOctop}
    Let $\{e_i\}$ be the standard basis of $\R^d$, $k\in [0,d]$,
    \begin{align*}
        \aaa= & \left\{\sum_{i=k+1}^{d}\delta_i e_i\right\}\cup\left\{e_1,\ldots,e_k\right\},\,\bb=\left\{\sum_{i=1}^{k}\delta_i e_i + e_j\right\}\cup\left\{\sum_{i=1}^{k}\delta_i e_i\right\}, \\
        & \text{ where } \delta_i\text{ range over }\left\{0, 1\right\} \text{ and }j\text{ over }[k+1,d].
    \end{align*}
    Here, $|\aaa|=2^{d-k}+k$ and $|\bb|=2^k (d-k+1)$.
\end{example}
We enumerated distinct sets with binary scalar products in dimensions up to 5, where `distinct' refers to an absence of linear isomorphism, and the results support Conjecture~\ref{generalisation}. The conjecture also holds for all sets that come from 2-level polytopes in dimensions up to 8.
